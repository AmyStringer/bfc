\documentclass{bioinfo}
\copyrightyear{2015}
\pubyear{2015}

\usepackage{amsmath}
\usepackage[ruled,vlined]{algorithm2e}
\newcommand\mycommfont[1]{\footnotesize\rmfamily{\it #1}}
\SetCommentSty{mycommfont}
\SetKwComment{Comment}{$\triangleright$\ }{}
\usepackage{natbib}

\bibliographystyle{apalike}


\begin{document}
\firstpage{1}

\title[Error Correction for Illumina Data]{Correcting Illumina sequencing errors for human data}

\author[Li]{Heng Li}

\address{Broad Institute, 75 Ames Street, Cambridge, MA 02142, USA}

\history{Received on XXXXX; revised on XXXXX; accepted on XXXXX}
\editor{Associate Editor: XXXXXXX}
\maketitle

\begin{abstract}
\section{Summary:} We present a new tool to correct sequencing errors in
Illumina data produced from whole-genome shotgun resequencing. It uses a
non-greedy algorithm and shows comparable performance and higher accuracy in an
evaluation that has the most complete collection of high-performance error
correctors so far.

\section{Availability and implementation:} https://github.com/lh3/bfc

\section{Contact:} hengli@broadinstitute.org
\end{abstract}

\section{Introduction}

Error correction is a process to fix sequencing errors on a sequence read
by using other overlapping reads that do not contain the errors. Many \emph{de
novo} assemblers, in particular short-read assemblers for large genomes, use
error correction to reduce the complexity of the assembly graph such that the
graph can be fitted to limited RAM. The most popular class of the error
correction algorithms is based on the $k$-mer spectrum~\citep{Pevzner:2001vn}.
It takes a list of trusted $k$-mers and attempts to find a sequence with
minimal corrections such that each $k$-mer on the corrected sequence is
trusted. Most fast implementations take a greedy approach. They make a
correction based on the local sequence context and do not revert the decision.
They may not find the sequence with the minimal corrections. We worried that
the greedy strategy might affect the accuracy given reads from a repeat-rich
diploid genome, so derived a new algorithm. It is optimal provided that we know
there is an error-free $k$-mer in the read.

\begin{methods}
\section{Methods}
Algorithm~1 is the key component of BFC. It defines a \emph{state}
of correction as a 4-tuple $(i,W,\mathcal{C},p)$, which consists of the
position $i$ of the preceding base, the last \mbox{($k$-1)--mer} $W$ ending at
$i$, the set $\mathcal{C}$ of previous corrected positions and bases (called a
\emph{solution}) up to $i$, and the penalty $p$ of solution $\mathcal{C}$. BFC
keeps all possible states in a priority queue $\mathcal{Q}$. At each iteration,
it retrieves the state $(i,W,\mathcal{C},p)$ with the lowest penalty $p$ and
adds a new state $(i+1,W[1,k-2]\circ a,\mathcal{C}',p')$ if $a$ is the read
base or $W\circ a$ is a trusted $k$-mer. If the first $k$-mer in $S$ is
error free and we remove line~2, this algorithm finds the optimal corrected
sequence.

It is possible to modify the algorithm to correct insertion and deletion
errors by augmenting the set of the ``next bases'' at line~1 to:
$$\mathcal{N}=\{(j,a)|j\in\{i-1,i\},a\in\{{\rm A},{\rm C},{\rm G},{\rm T}\}\}\cup\{(i,\epsilon)\}$$
In this set, $(i,a)$ substitutes a base at position $i$, $(i,\epsilon)$ deletes
the base and $(i-1,a)$ inserts a base $a$ before $i$. We have not
implemented this insertion/deletion-aware algorithm because such
errors are rare in Illumina data.

The worse-case time complexity of Algorithm~1 is exponential in the length of
the read. In implementation, we skip line~3 if the read base is Q20 and the
$k$-mer ending at it is trusted, or if five bases or two Q20 bases have been
corrected in the last 10bp window by default. The heuristics reduce the
search space. If BFC still takes too many iterations before finding an optimal
solution, we stop the search and mark the read not correctable.

Given a read, BFC finds the longest substring on which each $k$-mer is trusted. It
then extends the substring to both ends of the read with Algorithm~1. If a read
does not contain any trusted $k$-mers, BFC exhaustively tests all $k$-mer
one hamming distance away from the first $k$-mer on the read to find a trusted
$k$-mer. It marks the read uncorrectable if this fails.

We provided two related implementations of Algorithm~1, BFC-bf and BFC-ht.
BFC-bf uses KMC2~\citep{kmc2} to get exact $k$-mers counts and then keeps
trusted $k$-mers in a bloom filter. BFC-ht uses a combination of bloom filter
and in-memory hash table to derive approximate $k$-mer
counts~\citep{Melsted:2011bh} and counts of $k$-mers consisting of Q20 bases.
We modified Algorithm~1 such that missing trusted high-quality $k$-mers incurs
an extra penalty. This supposedly helps to correct systematic sequencing errors
which are recurrent but have lower base quality.

The $k$-mer size is an important parameter. Longer $k$-mers resolve more
repeats, while shorter $k$-mers increase $k$-mer depth and thus add power in
regions with low sequencing coverage. To take advantage of both long and short
$k$-mers, we implemented a refinement mode. We could apply BFC to the output
of a previous run and only correct reads that are previously uncorrectable or 
still contain untrusted $k$-mers after correction. Typically we use a shorter
$k$-mer in the second round and reject the correction if the corrected read
contains a $k$-mer with excessively high occurrences or contains more untrusted
$k$-mers.

%To take advantage of multiple CPU cores, we implemented a blocked bloom
%filter~\citep{DBLP:conf/wea/PutzeSS07} with a 1-byte spin lock for every
%63-byte block to prevent concurrent modifications to the same block.  For human
%data, we used an array of 16 million (=$2^{24}$) hash tables to keep $k$-mer
%counts. Each hash table has a spin lock. With a good hash function, locking is
%infrequent. We also put computing and I/O on different threads.  The strategy
%is particularly useful when I/O is slow.

\begin{algorithm}[ht]
\DontPrintSemicolon
\footnotesize
\KwIn{K-mer size $k$, set $\mathcal{H}$ of trusted $k$-mers, and one string $S$}
\KwOut{Set of corrected positions and bases changed to}
\BlankLine
\textbf{Function} {\sc CorrectErrors}$(k, \mathcal{H}, S)$
\Begin {
	$\mathcal{Q}\gets${\sc HeapInit}$()$\Comment*[r]{$\mathcal{Q}$ is a priority queue}
	{\sc HeapPush}$(\mathcal{Q}, (k-2, S[0,k-2], \emptyset, 0))$\Comment*[r]{0-based strings}
	\While{$\mathcal{Q}$ is not empty} {
		$(i, W, \mathcal{C}, p)\gets${\sc HeapPopBest}$(\mathcal{Q})$\Comment*[r]{current best state}
		$i\gets i+1$\;
		\lIf{$i=|S|$} { {\bf return} $\mathcal{C}$ \Comment*[f]{reaching the end of $S$}}
		\nl$\mathcal{N}\gets\{(i,{\rm A}),(i,{\rm C}),(i,{\rm G}),(i,{\rm T})\}$\Comment*[r]{set of next bases}
		\ForEach (\Comment*[f]{try all possible next bases}) {$(j,a)\in\mathcal{N}$} {
			$W'\gets W\circ a$\Comment*[r]{``$\circ$'' concatenates strings}
			\uIf (\Comment*[f]{no correction}) {$i=j$ {\bf and} $a=S[j]$} {
				\eIf (\Comment*[f]{good read base; no penalty}) {$W'\in\mathcal{H}$} {
					{\sc HeapPush}$(\mathcal{Q}, (j,W'[1,k-1],\mathcal{C}, p))$\;
				} (\Comment*[f]{bad read base; penalize}) {
					\nl{\sc HeapPush}$(\mathcal{Q}, (j,W'[1,k-1],\mathcal{C}, p+1))$\;
				}
			} \ElseIf (\Comment*[f]{make a correction with penalty}) {$W'\in \mathcal{H}$} {
				\nl{\sc HeapPush}$(\mathcal{Q}, (j,W'[1,k-1],\mathcal{C}\cup\{(j,a)\},p+1))$\;
			}
		}
	}
}
\caption{Error correction for one string in one direction}
\end{algorithm}

\end{methods}

\section{Results and Discussions}

We evaluated BFC along with BBMap-34.38 (\mbox{http://bit.ly/bbMap}),
BLESS-v0p23~\citep{Heo:2014aa}, Bloocoo-1.0.4~\citep{Drezen:2014aa},
fermi2-r175~\citep{Li:2012fk}, Lighter-20140123~\citep{Song:2014aa},
Musket-1.1~\citep{Liu:2013ac} and SGA-0.9.13~\citep{Simpson:2012aa} on real
data (Table~1). We ran the tools on a Linux server with 20 cores of Intel
E5-2660 CPUs and 128GB RAM. Precompiled binaries are available through
http://bit.ly/biobin and the command lines were included in the BFC source code
package (http://bit.ly/bfc-eval).  Notably, BLESS only works with uncompressed
files. The rest of tools were provided with gzip'd files as input. We have also
tried AllPaths-LG~\citep{Gnerre:2011ys}, Fiona~\citep{Schulz:2014aa} and
Trowel~\citep{Lim:2014aa}, but they require more RAM than our machine.
QuorUM-1.0.0~\citep{Zimin:2013aa} always trims reads, making it hard to be
compared to others which keep full-length reads.

\begin{table}[t]
\processtable{Performance of error correction}
{\footnotesize
\begin{tabular}{lcrrccrr}
\toprule
Prog.     & $k$ & Time  & RAM   & Perfect&Chim.& Better & Worse \\
\midrule
raw data  & --  & --    & --    & 2.40M  & 12.4k  & --     & -- \\
BBMap     & 31  &{\bf 3h22m}&33.0G& 2.78M& 12.4k  & 505k   & 19.2k \\
BFC-ht    & 37  & 6h33m & 63.5G & 3.04M  & 12.5k  & 830k   & 9.7k \\
BFC-ht    & 55  & 5h51m & 67.9G & 3.05M  &11.7k& 830k   &{\bf 9.0k}\\
BFC-ht    &55/33& 9h18m & 67.9G &{\bf 3.07M}&11.8k&{\bf 861k}&9.5k\\
BFC-bf    & 31  & 7h32m & 23.3G & 3.01M  & 13.1k  & 783k   & 9.2k \\
BFC-bf    & 55  & 4h41m & 23.3G & 3.05M  &11.8k& 819k   & 11.4k \\
BLESS     & 31  & 6h31m & 22.3G & 2.91M  & 13.1k  & 674k   & 20.8k \\
BLESS     & 55  & 5h09m & 22.3G & 3.01M  &{\bf 11.5k}& 775k& 10.3k \\
Bloocoo   & 31  & 5h52m &{\bf 4.0G}&2.88M& 14.1k  & 764k   & 31.5k  \\
Fermi2    & 29  &17h14m & 64.7G & 3.00M  & 17.7k  & 849k   &42.8k \\
Lighter   & 31  & 5h12m & 13.4G & 2.98M  & 13.0k  & 756k   & 30.1k  \\
Musket    & 27  &21h33m & 77.5G & 2.94M  & 22.5k  & 790k   & 36.3k  \\
SGA       & 55  &48h40m & 35.6G & 3.01M  & 12.1k  & 755k   & 12.8k  \\
\botrule
\end{tabular}}{4.45 million pairs of $\sim$150bp reads were downloaded from
BaseSpace, under the sample ``NA12878-L7'' of project ``HiSeq X Ten: TruSeq
Nano (4 replicates of NA12878)'', and were corrected together. On a subset of
two million randomly sampled read pairs, the original and the corrected
sequences were mapped to hs37d5 (http://bit.ly/GRCh37d5) with
BWA-MEM~\citep{Li:2013aa}.  A read is said to become \emph{better} (or
\emph{worse}) if the best alignment of the corrected sequence has more (or
fewer) identical bases to the reference genome than the best alignment of the
original sequence. The table gives $k$-mer size (maximal size used for Bloocoo,
fermi2, Lighter and Musket), the wall-clock \emph{time} when 16 threads are
specified if possible, the peak \emph{RAM} measured by GNU time, number of
corrected reads mapped \emph{perfectly}, number of \emph{chimeric} reads,
number of corrected reads becoming \emph{better} and the number of reads
becoming \emph{worse} than the original reads. For each metric, the best tool
is highlighted in the bold fontface.}
\end{table}

As is shown in the table, BBMap is the fastest. BFC, BLESS, Bloocoo and Lighter
are comparable in speed. Bloocoo is the most lightweight. Other bloom filter
based tools, BFC-bf, BLESS and Lighter, also have a small memory footprint.
Most evaluated tools have broadly comparable accuracy. Both BFC implementations
are marginally better, correcting more reads with few overcorrections when a
similar $k$-mer length is in use. Two-round correction is even better, though
it is slower as we have to run BFC-ht twice. We should note that it is possible
to tune the balance between accuracy, speed and memory for each tool. We have
not fully explored all the options.

On this data set, KMC2 counted 3.05 billion unique 55-mers occurring twice or
more, but only 2.84 billion 31-mers, 6.8\% less. This probably explains why
using longer $k$-mers consistently leads to higher accuracy. However, when
we ran the fermi 



When a single $k$-mer size is in use, longer $k$-mers appear better. This is
probably because longer $k$-mers resolve more repeat sequences.


However, long $k$-mers lead to reduced $k$-mer depth,
which is magnified by the increased chance of having an error on a long $k$-mer.
The reduced depth especially in low-coverage regions may affect the contiguity
of an assembly. We ran the fermi2 assembler~\citep{Li:2012fk} on the 37-mer
and 55-mer BFC-ht corrected reads. The 37-mer correction leads to longer
aligned N50 although the 55-mer correction looks better in the table.

\section*{Acknowledgement}
\paragraph{Funding\textcolon} NHGRI U54HG003037; NIH GM100233

\bibliography{bfc}
\end{document}
