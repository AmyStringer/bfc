\documentclass{bioinfo}
\copyrightyear{2015}
\pubyear{2015}

\usepackage{amsmath}
\usepackage[ruled,vlined]{algorithm2e}
\usepackage{natbib}

\bibliographystyle{apalike}


\begin{document}
\firstpage{1}

\title[Error Correction for Illumina Data]{Correcting Illumina sequencing errors for human data}

\author[Li]{Heng Li}

\address{Broad Institute, 75 Ames Street, Cambridge, MA 02142, USA}

\history{Received on XXXXX; revised on XXXXX; accepted on XXXXX}
\editor{Associate Editor: XXXXXXX}
\maketitle

\begin{abstract}
\section{Summary:}

\section{Availability and implementation:} https://github.com/lh3/bfc

\section{Contact:} hengli@broadinstitute.org
\end{abstract}

\section{Introduction}

\begin{methods}
\section{Methods}
The BFC error correction algorithm is a $k$-mer spectrum
algorithm~\citep{Pevzner:2001vn}, which takes a list of trusted $k$-mers and
attempts to find a corrected sequence such that each $k$-mer on the corrected
sequence is trusted. There are multiple ways to implement this basic idea.
Algorithm~1 gives the BFC implementation. It uses a 4-tuple as a state of
correction. A 4-tuple $(i,W,\mathcal{C},p)$ consists of the position $i$ of the
last visited base, the preceding \mbox{($k$-1)--mer} $W$ ending at $i$, the set
$\mathcal{C}$ of previous corrected positions and bases (called a
\emph{solution}) and the penalty calculated for solution $\mathcal{C}$. In the
pseudocode, either a correction or an untrusted $k$-mer incurs a penalty 1
(line~2 and~3). BFC keeps all possible states in a priority queue
$\mathcal{Q}$. At each iteration, it retrieves the state $(i,W,\mathcal{C},p)$
with the lowest penalty $p$ and adds a new state $(i+1,W[1,k-1]\circ
a,\mathcal{C}',p')$ if $a$ is the read base or $W\circ a$ is a trusted $k$-mer
($\circ$ is the string concatenation operator). We find the optimal solution
when we come to the end of the read (line~1). This algorithm may change a base
on a trusted $k$-mer if doing this leads to more trusted $k$-mers later.  Some
algorithms such as BLESS~\citep{Heo:2014aa} and Lighter~\citep{Song:2014aa}
apparently do not backtrack to change a wrong base assignment at an early
iteration.

Algorithm~1 is exponential in the length of the read in the worst case.  In
implementation, we skip line~3 if the read base is Q20 and the $k$-mer ending
at it is trusted, or if five bases or two Q20 bases have been corrected in the
last 10bp window by default. These heuristics reduce the search space.  If BFC
still takes too many iterations before finding an optimal solution, we stop the
search and mark the read not correctable.

\begin{algorithm}[ht]
\DontPrintSemicolon
\footnotesize
\KwIn{K-mer size $k$, trusted k-mer hash table $\mathcal{H}$ and one string $S$}
\KwOut{Set of corrected positions and bases changed to}
\BlankLine
\textbf{Function} {\sc CorrectErrors}$(k, \mathcal{H}, S)$
\Begin {
	$\mathcal{Q}\gets${\sc HeapInit}$()$\;
	{\sc HeapPush}$(\mathcal{Q}, (k-2, S[0,k-2], \emptyset, 0))$\;
	\While{$A$ is not empty} {
		$(i, W, \mathcal{C}, p)\gets${\sc HeapPopBest}$(\mathcal{Q})$\;
		$i\gets i+1$\;
		\nl\lIf{$i=|S|$} { {\bf return} $\mathcal{C}$ }
		\For{$a\in\{{\rm A},{\rm C},{\rm G},{\rm T}\}$} {
			$(p',W')\gets (p,W\circ a)$\;
			\uIf {$a=S[i]$} {
				\nl\lIf{$W'\not\in \mathcal{H}$} {$p'\gets p'+1$}
				{\sc HeapPush}$(\mathcal{Q}, (i,W'[1,k-1],\mathcal{C}, p'))$\;
			} \ElseIf{$W'\in \mathcal{H}$}{
				\nl{\sc HeapPush}$(\mathcal{Q}, (i,W'[1,k-1],\mathcal{C}\cup\{(i,a)\},p'+1))$\;
			}
		}
	}
}
\caption{Error correction for one string in one direction}
\end{algorithm}

\end{methods}

\section{Results and Discussions}
\begin{table}[ht]
\processtable{Performance of error correction for 4 million human reads}
{\footnotesize
\begin{tabular}{lrccccrc}
\toprule
Prog.     & Time & RAM   & Perfect & Chimeric & Better & Worse & Ambi. \\
\midrule
BFC       & 7.4h & 63.5G & {\bf 3.04M} & {\bf 12.5k} & {\bf 820k}   & {\bf 7.1k}  & {\bf 14.3k} \\
BLESS     & 6.5h & 22.3G & 2.91M   & 13.1k    & 666k   & 15.2k & 14.8k \\
Fermi2    &17.2h & 64.7G & \\
Lighter   &{\bf 3.2h}&{\bf 13.4G}&2.98M&13.0k & 743k   & 22.4k & 21.8k \\
Musket    &\\
\botrule
\end{tabular}}{Column meaning: the wall-clock \emph{time} when 16 threads
are specified if possible, the peak \emph{RAM} measured by GNU time, number of
reads mapped \emph{perfectly} to hs37d5 by BWA-MEM, number of \emph{chimeric}
reads, number of corrected reads \emph{better} than the original reads, number
of reads \emph{worse} than the original and number of corrected reads with some
aspects better but some worse than the original reads (column \emph{Ambi}).}

\end{table}

\section*{Acknowledgement}
\paragraph{Funding\textcolon} NHGRI U54HG003037; NIH GM100233

\bibliography{bfc}
\end{document}
